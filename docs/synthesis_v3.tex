\documentclass{article}
\usepackage{amsmath}
\usepackage{amssymb}
\usepackage{xcolor}
\newcommand{\jg}[1] {{\color{purple}{\bf{JG COMMENT: #1}}}}
\begin{document}
\title{Organization Structure with Information Synthesis}
\maketitle

% This document serves three purposes:
% \begin{enumerate}
% \item To provide a general and formal specification of our model.  
% \item To show how our model is a generalization of Calvó‐Armengol,
%   Antoni, Joan Martí, and Andrea Prat. "Communication and influence."
%   Theoretical Economics 10.2 (2015): 649-690. 
% \item To translate the purely mathematical specification to a neural
%   network that can be trained to solve to model
% \end{enumerate}
% However, to make things clearer, I will start with the review of the
% Communication and Influence (CI) model and then develop our synthesis
% model and then compare the two.  To make things clear and precise, I
% will avoid the terms ``node'', ``edge'' and any other graph-theoretic
% concept until they are precisely defined.  

% \section{CI Model}
% I simplify their model to ease up on notation.  See the exact paper
% for the full specification.  There are $n$ agents.  Each agent
% receives an independent signal 
% $\theta_i ~ \mathcal{N}(0, \sigma)$.  Each agent $i$ communicates with
% each other agent $j \neq i$.  The signal agent $j$ receives from agent
% $i$ is given by 
% \begin{equation}
% y_{ij} = \theta_i + \epsilon_{ij}, + \eta_{ij}
% \end{equation}
% where
% \begin{align}
% \epsilon_{ij} ~ \mathcal{N}(0, 1/r_{ij}) \nonumber \\
% \eta_{ij} ~ \mathcal{N}(0, 1/p_{ij}) \nonumber \\
% \end{align}
% The interpretation is that $\epsilon{ij}$ is the ``noise associated
% with active communication (preparing a presentation, writing a report,
% hosting a visit, etc)'' and $\eta_{ij}$ is the noise associated with
% passive communication (listening to a presentation, reading a report,
% visiting a plant, etc).''  

% After receiving a signal, agents choose action $a_i$ in $(-\infty,
% \infty)$.  Agent $i$'s utility is given by:
% \begin{equation}
% u_i = -((a_i - \theta_i)^2  + \sum_{j \neq i}(a_i -a_j)^2
% \end{equation}
% Intuitively, agent $i$ wants to be close to the environment signal it
% received and also be close to other agent's actions, which also want
% to be close to the signal they receive.  The welfare function is just
% then $\sum_iu_i$.  Note that that the welfare function is just a
% function over the joint state of the environment (the $\theta$ terms)
% and the ``states'' of the agents (the $a$ terms).  

% However, the $r_{ij}$ and $p_{ij}$ terms are set endogenously.  That
% is, agents invest in reducing the noise in their message as well as
% the noise in the messages they receive.  These reductions are costly
% and the utility function (and associated) social welfare function are
% expanded to include these costs.  

% Without talking about how they solve the model (which is relatively
% unimportant to us at this point), let's tease out some of the main
% elements of the model.

% \begin{enumerate}
% \item All to all communication
% \item All agents' state contribute to the social welfare function
% \item No information synthesis
% \item Agent's main choice variable is to set the signal-to-noise ratio
%   in their communication
% \end{enumerate}

% It is the last element that I want to focus on.  In this model, the
% agents do not choose their message but instead simply choose how to
% reduce the noise.  In the specifications we have been talking about,
% agents choose their message and how to process incoming information,
% possibly in the presence of noise.  However, what I present below will
% be a combination of both.  In other words, agents will optimally
% choose how to optimally combine their input to outputs as well as
% simultaneously optimize the signal-to-noise ratio of their incoming
% and outgoing messages.  

\section{Synthesis model}
This is the full mathematical specification of the model.  It is
\emph{not} graph theoretic.    The model is in the
spirit of Calvó‐Armengol, Antoni, Joan Martí, and Andrea
Prat. ``Communication and influence.''  Theoretical Economics 10.2
(2015): 649-690, but is far broader (more information about that model
is in the commented out version of this tex).  In this write-up, I
sometimes make assumptions for notational convenience but note that
such assumptions can be relaxed.

The model contains $N$ agents indexed by $i=1...N$.  There are also
$K$ random variables representing a different part of the state of the
world or the environment.  For simplicity, assume that each element of
the environment is independent so that
\begin{equation}
  \theta_k \sim f_k
\end{equation}
where $\theta_k$ is the state of random variable $k$ and $f_k$ is a
probability density/distribution function.  In general $K\neq N$.  For
simplicity, assume that each of $\theta_k$ is one-dimensional.
Allowing $\theta$ to be multi-dimensional as well as allowing each
$\theta$ to be correlated would be a trivial extension at the expense
of more notation.

Each agent $i$ takes an action.  Each agent also sends and receives
messages.  We assume that agent $i$ broadcasts.  That is, whatever
agent $i$ says can be heard by all other agents, if they choose to do
so (think mass emails). 

% First, I will discuss how agents send messages, given their state.
% Then I will discuss how agents receive messages.  Then I will discuss
% how agents observe the environment.  Finally, I will discuss how
% agents set their internal state.
First, I will discuss how agents receive messages and observe the
environment, then talk about how they choose their action  and finally
discuss how they send messages. 

\subsection{Receiving Observations}
In this  section, we will discuss the observations an agent receives.
Let $E_i$ be a $1\times G_i$ row vector.  Later we will discuss the
elements of $E_i$ but intuitively, $E_i$ represents observations from
the environment and whatever any of the other agents say.  Then, the
observations of agent $i$ is given by
\begin{equation}
  \mathbf{O_{i}} = \gamma_i(\mathbf{E_i} + \mathbf{W_i}\odot
  \mathcal{N}^e_{i}(0, \Sigma^e_{i}))
\end{equation}
where $W_i$ is a $1\times G_i$ row vector of the form $(\frac{1}{w_1},\frac{1}{w_1}...\frac{1}{w_{G_i}})$
and $\odot$ represents element-wise multiplication.  $\gamma_i$ is a
function from $\mathbb{R}^{G_i} \rightarrow \mathbb{R}^{G_i}$.  For
now, we can just think of $\gamma_i$ as the identity 
function but later, we might want it to be non-linear.  The term
$\mathcal{N}^e_{i}(0, \Sigma^e_{i})$ represents a $1\times G_i$ vector
of normal random variables with standard deviation $\Sigma^e_{i}$.

Intuitively, agent $i$'s observations, $O_i$, are simply observations
of the environment or the physical world \emph{and} messages sent by
other players, both of which are corrupted by noise.  


\subsection{Choosing Actions}


Given what agent $i$ observes, $O_i$, agent $i$'s action is given by
\begin{equation}
  \mathbf{A_{i}} = \alpha_i(O_iX_i)
\end{equation}
where $X_i$ is a $G_i\times D_i$ matrix.  This means that $A_i$ is a
$1\times D_i$ row vector.  Intuitively, $D_i$ represents the number of
different actions that agent $i$ can take.  That is to say, if $D_i$
is 2, then agent $i$'s action space is two dimensional and thus agent
$i$ can take $2$ actions.  $\alpha$ is a function but for now think of
it as the identity.  

\subsection{Sending Messages}
Given agent $i$'s observation $O_i$, what agent $i$ says is given by
\begin{equation}
M_i = \beta(O_i\Omega_i + \mathcal{N}^m_{i}(0, \Sigma^m_{i}))
\end{equation}
where $\Omega_i$ is a $G_i\times F_i$ matrix and $\mathcal{N}^m_{i}(0,
\Sigma^m_{i})$ is a $1\times F_i$ vector of normal random
variables.  That means that $M_i$ is a $1\times F_i$ vector.
Intuitively, $F_i$ represents the number of distinct things that agent
$i$ can say.  Intuitively, agent $i$ can determine what it says by
setting elements of $\Omega_i$ and the magnitude of those elements
determine the signal to noise ratio.  $\beta$ is a function but for
now, think of it as the identity.

\subsection{Composition of $E_i$}
So what exactly is $E_i$?  Simple, $E_i$ is just the concatenation of
the environment and what all other agents $j<i$ say.  So, for agent
$1$, $E_i$ is a $1\times K$ vector.  For agent 2, $E_i$ is a $1\times
K + F_1$ vector.  For agent $3$, $E_i$ is a $1\times K + F_1 +F_2$ 
vector.  In general, for agent $i$, $E_i$ is a $1\times K +
\sum_{j<i}F_i$ vector.



\section{Optimization Problem}
As shorthand, let $\Omega = [\Omega_1, \Omega_2...\Omega_N]$, $X=[X_1,
X_2...X_N]$, $W=[W_1, W_2...W_n]$ and $A=[A_1, A_2...A_N]$. 

Finally, we can write the welfare function as:
\begin{equation}
  F = U(\theta, A) -  ||\Omega||^d - ||W||^d
\end{equation}
where $||\mathbf{q}||^d$ represents the $L^d$ norm of matrix
$\mathbf{q}$ and $U$ is some function of the environment and the
states of the nodes.  The important element to note about the welfare
function is that it does \emph{not} explicitly depend on $X$, which
represents the agent's internal computation on how to combine inputs
to outputs.  Of course, $X$ implicitly enters the utility function
through $U$ (since $A$ is a function of $X$), but there is no cost
associated with the weights.  The optimization problem then becomes
\begin{equation}
  \max_{\Omega, X, W}F
\end{equation}

\subsection{Signal-to Noise Ratio}
The main element of this model is that there is a \textbf{clear direct
  relationship} between signal-to-noise ratio and communication cost.
The higher the signal-to-noise ratio, the higher the penalization in
the welfare function.  \jg{Question to Milo: Do you see why we need to
  penalize for the magnitude of $\Omega$?.  If not, agent $i$ can just
  blow up his message so that the noise is irrelevant and all agents
  listening to him can just shrink it down.}


\section{Real-World Interpretation of Variables}


\begin{table}[h!]
\centering
\begin{tabular}{|p{1in}|p{1.5in}|p{3in}|}
\hline
Item       & Type            & Interpretation                                                                                           \\
$N$          & parameter       & Number of Agents in an Organization                                                                      \\
$K$          & parameter       & Number of random variables in the environment                                                            \\
$\theta_k$ & Random Variable & Environment External to the Organization                                                                 \\
$f_k$      & parameter       & Distribution of environment variable $k$                                                                 \\
$E_i$      & Variable        & All of the things (environment and other people) that agent $i$ could possible listen t.o                \\
$W_i$      & Variable        & A matrix that represents how agent $i$ allocates its attention to everything it could possibly listen to \\
$O_i$      & Variable        & What agent $i$ "observes" after it decides where to allocate its attention                               \\
$A_i$      & Variable        & The action that agent $i$ takes                                                                          \\
$X_i$      & Variable        & A variable that represents how agent $i$ combines what it observes $O_i$ to determine what it does.      \\
$M_i$      & Variable        & What agent $i$ says                                                                                      \\
$\Omega_i$ & Variable        & How agent $i$ transforms its observations to
                               what it says.
  \\ \hline
\end{tabular}
\end{table}



\end{document} 
